\section{\texttt{domain\_class}}

\subsection{Description}
This class defines 2-dimensional spatial domains in cartesian and/or spherical coordinates used for field calculations of all multipolar fields. The planes are always centered around origo.
It supports customizable size, meaning that it can be defined in terms of wavelength multiples. Furthermore, resolution and plane selection are customizable.

\subsection{Initialization}
Creating an instance of a \texttt{domain} object initializes the desired planes as 2D \texttt{numpy} arrays with a specified resolution.

\begin{verbatim}
domain(size, points = 150, planes = {'xy', 'xz'})
\end{verbatim}

\begin{description}[leftmargin=3cm]
    \item[size:] float — width of the domain from center to edge in microns.
    \item[points:] int — number of points along each axis in the grid. Default is 150.
    \item[planes:] set — must be any of \texttt{'xy'}, \texttt{'xz'} or \texttt{'yz'}. Defaults to \texttt{\{'xy', 'xz'\}}.
\end{description}

\subsection{Methods}
\begin{description}[leftmargin=3.5cm]
    \item[\texttt{check\_params(size, planes, points)}] \hfill
    Ensures \texttt{size} and \texttt{points} are positive numbers and \texttt{planes} are valid. Called directly upon initialization.

    \item[\texttt{cart\_grid()}] \hfill \\
    Defines 2D cartesian grids based on inputs.\\
    Returns a dict with planes as keys and corresponding \texttt{numpy.meshgrid()} 2D arrays as values with Cartesian coordinates. For example, the \texttt{'xy'} key maps to \texttt{numpy.array([x, y])}.

    \item[\texttt{cart\_coords()}] \hfill \\
    Returns a dict with keys corresponding to planes and values being 3D \texttt{numpy} arrays \texttt{[x, y, z]}. The missing coordinate for each plane is filled with zeros.\\
    Used for plotting on cartesian grids rather than spherical.

    \item[\texttt{spherical\_grid()}] \hfill \\
    Returns a dict with identical keys (planes). The values are 3D \texttt{numpy} arrays with radial, polar, and azimuthal coordinates of the plane, defined as

    \begin{align*}
        r &= \sqrt{x^2 + y^2 + z^2} \\
        \theta &= \arccos\left(\frac{z}{r}\right) \quad \in [0, \pi] \\
        \phi &= 
        \begin{cases}
            \arctan2(y, x) & \text{for xy and xz planes} \\
            \arctan2(x, y) + \pi/2 & \text{for yz plane}
        \end{cases}
    \end{align*}

    \item[\texttt{show\_coord(coord, spherical\_grids)}] \hfill
    Visualizes the spherical coordinate values. \texttt{coord} is either \texttt{'r'}, \texttt{'theta'}, or \texttt{'phi'}, and \texttt{spherical\_grids} is the dict returned by \texttt{spherical\_grid()}. Plots the requested coordinate in all defined planes using \texttt{matplotlib}.
\end{description}


